\documentclass[]{article}
\usepackage{lmodern}
\usepackage{amssymb,amsmath}
\usepackage{ifxetex,ifluatex}
\usepackage{fixltx2e} % provides \textsubscript
\ifnum 0\ifxetex 1\fi\ifluatex 1\fi=0 % if pdftex
  \usepackage[T1]{fontenc}
  \usepackage[utf8]{inputenc}
\else % if luatex or xelatex
  \ifxetex
    \usepackage{mathspec}
  \else
    \usepackage{fontspec}
  \fi
  \defaultfontfeatures{Ligatures=TeX,Scale=MatchLowercase}
\fi
% use upquote if available, for straight quotes in verbatim environments
\IfFileExists{upquote.sty}{\usepackage{upquote}}{}
% use microtype if available
\IfFileExists{microtype.sty}{%
\usepackage{microtype}
\UseMicrotypeSet[protrusion]{basicmath} % disable protrusion for tt fonts
}{}
\usepackage[margin=1in]{geometry}
\usepackage{hyperref}
\hypersetup{unicode=true,
            pdftitle={Report\_SN\_Summaries.R},
            pdfauthor={rstudio},
            pdfborder={0 0 0},
            breaklinks=true}
\urlstyle{same}  % don't use monospace font for urls
\usepackage{color}
\usepackage{fancyvrb}
\newcommand{\VerbBar}{|}
\newcommand{\VERB}{\Verb[commandchars=\\\{\}]}
\DefineVerbatimEnvironment{Highlighting}{Verbatim}{commandchars=\\\{\}}
% Add ',fontsize=\small' for more characters per line
\usepackage{framed}
\definecolor{shadecolor}{RGB}{248,248,248}
\newenvironment{Shaded}{\begin{snugshade}}{\end{snugshade}}
\newcommand{\KeywordTok}[1]{\textcolor[rgb]{0.13,0.29,0.53}{\textbf{#1}}}
\newcommand{\DataTypeTok}[1]{\textcolor[rgb]{0.13,0.29,0.53}{#1}}
\newcommand{\DecValTok}[1]{\textcolor[rgb]{0.00,0.00,0.81}{#1}}
\newcommand{\BaseNTok}[1]{\textcolor[rgb]{0.00,0.00,0.81}{#1}}
\newcommand{\FloatTok}[1]{\textcolor[rgb]{0.00,0.00,0.81}{#1}}
\newcommand{\ConstantTok}[1]{\textcolor[rgb]{0.00,0.00,0.00}{#1}}
\newcommand{\CharTok}[1]{\textcolor[rgb]{0.31,0.60,0.02}{#1}}
\newcommand{\SpecialCharTok}[1]{\textcolor[rgb]{0.00,0.00,0.00}{#1}}
\newcommand{\StringTok}[1]{\textcolor[rgb]{0.31,0.60,0.02}{#1}}
\newcommand{\VerbatimStringTok}[1]{\textcolor[rgb]{0.31,0.60,0.02}{#1}}
\newcommand{\SpecialStringTok}[1]{\textcolor[rgb]{0.31,0.60,0.02}{#1}}
\newcommand{\ImportTok}[1]{#1}
\newcommand{\CommentTok}[1]{\textcolor[rgb]{0.56,0.35,0.01}{\textit{#1}}}
\newcommand{\DocumentationTok}[1]{\textcolor[rgb]{0.56,0.35,0.01}{\textbf{\textit{#1}}}}
\newcommand{\AnnotationTok}[1]{\textcolor[rgb]{0.56,0.35,0.01}{\textbf{\textit{#1}}}}
\newcommand{\CommentVarTok}[1]{\textcolor[rgb]{0.56,0.35,0.01}{\textbf{\textit{#1}}}}
\newcommand{\OtherTok}[1]{\textcolor[rgb]{0.56,0.35,0.01}{#1}}
\newcommand{\FunctionTok}[1]{\textcolor[rgb]{0.00,0.00,0.00}{#1}}
\newcommand{\VariableTok}[1]{\textcolor[rgb]{0.00,0.00,0.00}{#1}}
\newcommand{\ControlFlowTok}[1]{\textcolor[rgb]{0.13,0.29,0.53}{\textbf{#1}}}
\newcommand{\OperatorTok}[1]{\textcolor[rgb]{0.81,0.36,0.00}{\textbf{#1}}}
\newcommand{\BuiltInTok}[1]{#1}
\newcommand{\ExtensionTok}[1]{#1}
\newcommand{\PreprocessorTok}[1]{\textcolor[rgb]{0.56,0.35,0.01}{\textit{#1}}}
\newcommand{\AttributeTok}[1]{\textcolor[rgb]{0.77,0.63,0.00}{#1}}
\newcommand{\RegionMarkerTok}[1]{#1}
\newcommand{\InformationTok}[1]{\textcolor[rgb]{0.56,0.35,0.01}{\textbf{\textit{#1}}}}
\newcommand{\WarningTok}[1]{\textcolor[rgb]{0.56,0.35,0.01}{\textbf{\textit{#1}}}}
\newcommand{\AlertTok}[1]{\textcolor[rgb]{0.94,0.16,0.16}{#1}}
\newcommand{\ErrorTok}[1]{\textcolor[rgb]{0.64,0.00,0.00}{\textbf{#1}}}
\newcommand{\NormalTok}[1]{#1}
\usepackage{graphicx,grffile}
\makeatletter
\def\maxwidth{\ifdim\Gin@nat@width>\linewidth\linewidth\else\Gin@nat@width\fi}
\def\maxheight{\ifdim\Gin@nat@height>\textheight\textheight\else\Gin@nat@height\fi}
\makeatother
% Scale images if necessary, so that they will not overflow the page
% margins by default, and it is still possible to overwrite the defaults
% using explicit options in \includegraphics[width, height, ...]{}
\setkeys{Gin}{width=\maxwidth,height=\maxheight,keepaspectratio}
\IfFileExists{parskip.sty}{%
\usepackage{parskip}
}{% else
\setlength{\parindent}{0pt}
\setlength{\parskip}{6pt plus 2pt minus 1pt}
}
\setlength{\emergencystretch}{3em}  % prevent overfull lines
\providecommand{\tightlist}{%
  \setlength{\itemsep}{0pt}\setlength{\parskip}{0pt}}
\setcounter{secnumdepth}{0}
% Redefines (sub)paragraphs to behave more like sections
\ifx\paragraph\undefined\else
\let\oldparagraph\paragraph
\renewcommand{\paragraph}[1]{\oldparagraph{#1}\mbox{}}
\fi
\ifx\subparagraph\undefined\else
\let\oldsubparagraph\subparagraph
\renewcommand{\subparagraph}[1]{\oldsubparagraph{#1}\mbox{}}
\fi

%%% Use protect on footnotes to avoid problems with footnotes in titles
\let\rmarkdownfootnote\footnote%
\def\footnote{\protect\rmarkdownfootnote}

%%% Change title format to be more compact
\usepackage{titling}

% Create subtitle command for use in maketitle
\newcommand{\subtitle}[1]{
  \posttitle{
    \begin{center}\large#1\end{center}
    }
}

\setlength{\droptitle}{-2em}

  \title{Report\_SN\_Summaries.R}
    \pretitle{\vspace{\droptitle}\centering\huge}
  \posttitle{\par}
    \author{rstudio}
    \preauthor{\centering\large\emph}
  \postauthor{\par}
      \predate{\centering\large\emph}
  \postdate{\par}
    \date{Mon Mar 4 22:18:42 2019}


\begin{document}
\maketitle

\begin{Shaded}
\begin{Highlighting}[]
\NormalTok{################ Process xlsx from Stuart}
\CommentTok{# Routines to generate reports for ECC and Southrepps using data from SEN.}
\CommentTok{# EAP 2019-03-04}
\CommentTok{# Assumes original files have been:}
\CommentTok{#   imported,}
\CommentTok{#   tidied,}
\CommentTok{#   saved in the tidy sub-directory for each location}
\CommentTok{#   file names will be structured}
\CommentTok{#     YYYY-MM-DD_Summary_SEN_Evaluation_XXX.csv, where XXX is a valid site code}
\CommentTok{#  }
\CommentTok{# The input files have the following columns }
\CommentTok{#   obs_datetime : date a time of the recording }
\CommentTok{#   filename : relates to the original wav/wac file generated by the SM2}
\CommentTok{#   species : as identified by the classifier}
\CommentTok{#   confidence_index : as identified by the classifier, was called "accuracy"}
\CommentTok{#   real_error : as calculated by the classifier following methor in Barre et. al}
\CommentTok{# Structure of values in the filename column:}
\CommentTok{#   XXX             : chr > 3 digit site code  SR2 = Southrepps (Dowlands), ECC == Eccls}
\CommentTok{#   _               : chr > separator}
\CommentTok{#   yyyymmdd        : num > date of recording (assigned by SM2) ISO format}
\CommentTok{#   _               : chr > separator}
\CommentTok{#   hhmmss          : num > time of recording (assigned by SM2 no DST correction applied. hours since midnight)}
\CommentTok{#   _               : chr > separator}
\CommentTok{#   NNN              : num > 3 digit number assigned by classifier, thought to be the call number in the recording file.}
\CommentTok{#}
\CommentTok{# Processing will be required to:}
\CommentTok{#   load data files into a data.frame}
\CommentTok{#   }
\CommentTok{#   }
\CommentTok{#}
\CommentTok{# Input file are located in the following locations:}
\CommentTok{#   ~/R-Test/intermed/ECC, and}
\CommentTok{#   ~/R-Test/intermed/SR2}
\CommentTok{#}
\CommentTok{# Output files will be writen to the following locations:}
\CommentTok{#   ~/R-Test/tidy/ECC, and}
\CommentTok{#   ~/R-Test/tidy/SR2}
\CommentTok{#}
\CommentTok{# Output files will have the following structure:}
\CommentTok{#   2019-02-12_SEN_Evaluation_XXX.csv : where XXX is either ECC or SR2 as relevant}
\CommentTok{#}
\NormalTok{########  CONFIG Follows ##########################}

\CommentTok{# Load required libraries}

\CommentTok{# Load tidyvers functions}
\CommentTok{#if (!require(tidyverse)) install.packages('tidyverse')}
\KeywordTok{library}\NormalTok{(tidyverse)}
\CommentTok{#library(readr)}
\CommentTok{#library(dplyr)}
\CommentTok{#library(purrr)}
\KeywordTok{library}\NormalTok{(lubridate)}
\KeywordTok{library}\NormalTok{(readxl) }\CommentTok{#Needed to process xlxs files}
\KeywordTok{library}\NormalTok{(knitr)}

\CommentTok{#Evaluation parameters}
\NormalTok{re_threshold <-}
\StringTok{  }\FloatTok{0.5} \CommentTok{#Change this value to set required accuracy cut-off.}
\CommentTok{#In practice 0.5 is applied by Stuart when agreegating records.}
\NormalTok{save_csv <-}
\StringTok{  }\OtherTok{TRUE} \CommentTok{#Change to FALSE if you don't want to create a new csv file}
\NormalTok{site_code <-}\StringTok{ "ECC"} \CommentTok{#See below for alternatives}
\NormalTok{input_file_pattern <-}\StringTok{ "*_Summary_SEN_Evaluation*"}

\CommentTok{#output_file_name <- "2019-02-28_SEN_Evaluation_SR2.csv"}

\CommentTok{#Directories  NB these are only vaid for AWS - RStudio - Server}
\NormalTok{d_home <-}\StringTok{ "~/R-Test/"}
\NormalTok{d_raw <-}\StringTok{  }\KeywordTok{paste}\NormalTok{(d_home, }\StringTok{"raw/"}\NormalTok{, site_code, }\StringTok{"/"}\NormalTok{, }\DataTypeTok{sep =} \StringTok{""}\NormalTok{)}
\NormalTok{d_intermed <-}\StringTok{ }\KeywordTok{paste}\NormalTok{(d_home, }\StringTok{"intermed/"}\NormalTok{, site_code, }\StringTok{"/"}\NormalTok{, }\DataTypeTok{sep =} \StringTok{""}\NormalTok{)}
\NormalTok{d_tidy <-}\StringTok{ }\KeywordTok{paste}\NormalTok{(d_home, }\StringTok{"tidy/"}\NormalTok{, site_code, }\StringTok{"/"}\NormalTok{, }\DataTypeTok{sep =} \StringTok{""}\NormalTok{)}
\NormalTok{d_output <-}\StringTok{   }\KeywordTok{paste}\NormalTok{(d_home, }\StringTok{"output/"}\NormalTok{, }\DataTypeTok{sep =} \StringTok{""}\NormalTok{)}


\CommentTok{# Site Specific Information}
\NormalTok{validsitecodes <-}\StringTok{ }\KeywordTok{c}\NormalTok{(}\StringTok{"SR2"}\NormalTok{, }\StringTok{"ECC"}\NormalTok{)}


\NormalTok{########  CODE Follows ##########################}
\CommentTok{# Check if we have a valid site code}
\ControlFlowTok{if}\NormalTok{ (}\OperatorTok{!}\NormalTok{(site_code }\OperatorTok\StringTok{ }\NormalTok{validsitecodes))  \{}
  \KeywordTok{stop}\NormalTok{(}\StringTok{"Invalid Site Code"}\NormalTok{)}
\NormalTok{\}}


\CommentTok{# Configure Environment & paths etc.}
\KeywordTok{setwd}\NormalTok{(d_home)}
\KeywordTok{getwd}\NormalTok{()}
\end{Highlighting}
\end{Shaded}

\begin{verbatim}
## [1] "/home/rstudio/R-Test"
\end{verbatim}

\begin{Shaded}
\begin{Highlighting}[]
\CommentTok{# Read the input file}
\NormalTok{tmp_SNclassifier_results <-}\StringTok{ }\KeywordTok{list.files}\NormalTok{(}
  \DataTypeTok{path =} \KeywordTok{as.character}\NormalTok{(d_tidy),}
  \DataTypeTok{pattern =}\NormalTok{ input_file_pattern,}
  \DataTypeTok{recursive =} \OtherTok{TRUE}\NormalTok{,}
  \DataTypeTok{full.names =} \OtherTok{TRUE}
\NormalTok{) }\OperatorTok
\StringTok{  }\KeywordTok{map_df}\NormalTok{(}\OperatorTok{~}\StringTok{ }\KeywordTok{read_csv}\NormalTok{(.))}
\end{Highlighting}
\end{Shaded}

\begin{verbatim}
## Parsed with column specification:
## cols(
##   obs_datetime = col_datetime(format = ""),
##   filename = col_character(),
##   species = col_character(),
##   confidence_index = col_double(),
##   real_error = col_double()
## )
\end{verbatim}

\begin{Shaded}
\begin{Highlighting}[]
\CommentTok{# Monthly Summary, results writen to tbl_mnlyStats}
\NormalTok{tbl_mnlyStats <-}\StringTok{ }\NormalTok{tmp_SNclassifier_results }\OperatorTok
\StringTok{  }\NormalTok{dplyr}\OperatorTok{::}\KeywordTok{filter}\NormalTok{(., real_error }\OperatorTok{>=}\StringTok{ }\NormalTok{re_threshold) }\OperatorTok
\StringTok{  }\KeywordTok{group_by}\NormalTok{(}\KeywordTok{year}\NormalTok{(}\KeywordTok{as.Date}\NormalTok{(obs_datetime, }\StringTok{"%Y-%m-%d"}\NormalTok{)),}
           \KeywordTok{month}\NormalTok{(}\KeywordTok{as.Date}\NormalTok{(obs_datetime, }\StringTok{"%Y-%m-%d"}\NormalTok{)),}
\NormalTok{           species) }\OperatorTok
\StringTok{  }\NormalTok{dplyr}\OperatorTok{::}\KeywordTok{summarise}\NormalTok{(}
    \DataTypeTok{count =} \KeywordTok{n}\NormalTok{(),}
    \DataTypeTok{max =} \KeywordTok{max}\NormalTok{(confidence_index),}
    \DataTypeTok{mean =} \KeywordTok{round}\NormalTok{(}\KeywordTok{mean}\NormalTok{(confidence_index), }\DecValTok{2}\NormalTok{),}
    \DataTypeTok{min =} \KeywordTok{min}\NormalTok{(confidence_index),}
    \DataTypeTok{std_dev =} \KeywordTok{round}\NormalTok{(}\KeywordTok{sd}\NormalTok{(confidence_index), }\DecValTok{2}\NormalTok{)}
\NormalTok{  )}
\KeywordTok{names}\NormalTok{(tbl_mnlyStats)[}\DecValTok{1}\NormalTok{] <-}\StringTok{ "Year"}
\KeywordTok{names}\NormalTok{(tbl_mnlyStats)[}\DecValTok{2}\NormalTok{] <-}\StringTok{ "Month"}
\NormalTok{tbl_mnlyStats <-}\StringTok{ }\KeywordTok{as.data.frame}\NormalTok{(tbl_mnlyStats)}


\CommentTok{#Now generate species summaries}
\NormalTok{species_found <-}\StringTok{ }\KeywordTok{unique}\NormalTok{(tbl_mnlyStats}\OperatorTok{$}\NormalTok{species)}
\KeywordTok{print}\NormalTok{(}\KeywordTok{paste}\NormalTok{(site_code, }\StringTok{"Evaluation by SN"}\NormalTok{))}
\end{Highlighting}
\end{Shaded}

\begin{verbatim}
## [1] "ECC Evaluation by SN"
\end{verbatim}

\begin{Shaded}
\begin{Highlighting}[]
\ControlFlowTok{for}\NormalTok{ (row }\ControlFlowTok{in} \DecValTok{1}\OperatorTok{:}\KeywordTok{length}\NormalTok{(species_found)) \{}
\NormalTok{  tmp_species <-}
\StringTok{    }\KeywordTok{filter}\NormalTok{(tbl_mnlyStats, species }\OperatorTok{==}\StringTok{ }\NormalTok{species_found[row])}
  \KeywordTok{print}\NormalTok{(knitr}\OperatorTok{::}\KeywordTok{kable}\NormalTok{(tmp_species))}
  
\NormalTok{\}}
\end{Highlighting}
\end{Shaded}

\begin{verbatim}
## 
## 
##  Year   Month  species    count    max   mean    min   std_dev
## -----  ------  --------  ------  -----  -----  -----  --------
##  2017       9  Barbar         1   0.95   0.95   0.95        NA
##  2017      10  Barbar         2   0.99   0.99   0.99      0.00
##  2017      11  Barbar         6   0.99   0.96   0.86      0.05
##  2018       4  Barbar         1   0.99   0.99   0.99        NA
##  2018       6  Barbar         2   0.99   0.96   0.93      0.04
##  2018       7  Barbar         4   0.99   0.97   0.95      0.02
##  2018       8  Barbar         5   0.98   0.86   0.45      0.23
## 
## 
##  Year   Month  species    count    max   mean    min   std_dev
## -----  ------  --------  ------  -----  -----  -----  --------
##  2017       9  Eptser        15   0.98   0.88   0.61      0.10
##  2017      10  Eptser        11   0.98   0.75   0.41      0.19
##  2018       4  Eptser         5   0.99   0.95   0.91      0.03
##  2018       6  Eptser         4   0.98   0.86   0.59      0.18
##  2018       7  Eptser       171   0.99   0.86   0.40      0.16
##  2018       8  Eptser       150   0.99   0.84   0.44      0.16
##  2018      10  Eptser         3   0.85   0.67   0.43      0.22
## 
## 
##  Year   Month  species    count    max   mean    min   std_dev
## -----  ------  --------  ------  -----  -----  -----  --------
##  2017       9  Myodau         1   0.92   0.92   0.92        NA
##  2017      10  Myodau         2   0.97   0.74   0.51      0.33
##  2017      11  Myodau         1   0.50   0.50   0.50        NA
##  2018       4  Myodau         1   0.70   0.70   0.70        NA
##  2018       6  Myodau         1   0.47   0.47   0.47        NA
##  2018       7  Myodau         1   0.50   0.50   0.50        NA
##  2018       8  Myodau         2   0.75   0.66   0.57      0.13
## 
## 
##  Year   Month  species    count    max   mean    min   std_dev
## -----  ------  --------  ------  -----  -----  -----  --------
##  2017       9  Nycnoc       987   0.99   0.94   0.61      0.07
##  2017      10  Nycnoc       994   0.99   0.93   0.61      0.09
##  2017      11  Nycnoc        12   0.99   0.79   0.62      0.15
##  2018       4  Nycnoc        19   0.99   0.93   0.71      0.07
##  2018       5  Nycnoc        11   0.99   0.91   0.69      0.10
##  2018       6  Nycnoc        33   0.99   0.90   0.68      0.10
##  2018       7  Nycnoc       818   0.99   0.90   0.61      0.10
##  2018       8  Nycnoc       877   0.99   0.89   0.61      0.11
##  2018      10  Nycnoc         6   0.97   0.88   0.74      0.10
## 
## 
##  Year   Month  species    count    max   mean    min   std_dev
## -----  ------  --------  ------  -----  -----  -----  --------
##  2017       9  Pipnat        13   0.87   0.72   0.47      0.13
##  2017      10  Pipnat       137   0.98   0.85   0.47      0.13
##  2017      11  Pipnat         4   0.95   0.89   0.79      0.07
##  2018       4  Pipnat        51   0.97   0.82   0.47      0.14
##  2018       5  Pipnat        31   0.98   0.85   0.55      0.12
##  2018       6  Pipnat       457   0.98   0.81   0.47      0.12
##  2018       7  Pipnat      1441   0.98   0.77   0.47      0.10
##  2018       8  Pipnat         3   0.84   0.79   0.75      0.05
##  2018      10  Pipnat         3   0.79   0.65   0.55      0.12
## 
## 
##  Year   Month  species    count    max   mean    min   std_dev
## -----  ------  --------  ------  -----  -----  -----  --------
##  2017       9  Pippip       615   0.99   0.91   0.37      0.11
##  2017      10  Pippip      3018   0.99   0.92   0.37      0.10
##  2017      11  Pippip       162   0.99   0.95   0.37      0.09
##  2017      12  Pippip         9   0.98   0.90   0.38      0.20
##  2018       4  Pippip        84   0.99   0.93   0.40      0.10
##  2018       5  Pippip        53   0.99   0.92   0.42      0.11
##  2018       6  Pippip       774   0.99   0.87   0.37      0.11
##  2018       7  Pippip     10492   0.99   0.94   0.37      0.07
##  2018       8  Pippip      9876   0.99   0.93   0.37      0.10
##  2018      10  Pippip       778   0.99   0.92   0.37      0.08
## 
## 
##  Year   Month  species    count    max   mean    min   std_dev
## -----  ------  --------  ------  -----  -----  -----  --------
##  2017       9  Pippyg        87   0.99   0.88   0.35      0.14
##  2017      10  Pippyg       923   0.99   0.93   0.32      0.09
##  2017      11  Pippyg        57   0.99   0.95   0.41      0.08
##  2017      12  Pippyg         2   0.98   0.97   0.96      0.01
##  2018       4  Pippyg        14   0.99   0.86   0.39      0.19
##  2018       5  Pippyg        18   0.99   0.92   0.63      0.12
##  2018       6  Pippyg       130   0.99   0.96   0.51      0.06
##  2018       7  Pippyg       743   0.99   0.92   0.29      0.12
##  2018       8  Pippyg       445   0.99   0.89   0.30      0.14
##  2018      10  Pippyg        46   0.98   0.88   0.34      0.13
## 
## 
##  Year   Month  species    count    max   mean    min   std_dev
## -----  ------  --------  ------  -----  -----  -----  --------
##  2017      10  Myonat         1   0.98   0.98   0.98        NA
##  2017      11  Myonat         1   0.99   0.99   0.99        NA
##  2018       5  Myonat         2   0.99   0.96   0.94      0.04
## 
## 
##  Year   Month  species    count    max   mean    min   std_dev
## -----  ------  --------  ------  -----  -----  -----  --------
##  2018       6  Nyclei         1   0.36   0.36   0.36        NA
##  2018       7  Nyclei         3   0.54   0.47   0.43      0.06
\end{verbatim}

\begin{Shaded}
\begin{Highlighting}[]
\NormalTok{#### }\AlertTok{NOTE} \AlertTok{NOTE} \AlertTok{NOTE}\NormalTok{ ####}
\CommentTok{#}
\CommentTok{# To generate a pdf report from this process  it is not possible to use }
\CommentTok{# the RStudio ctr-K short cut as this throws a number of errors.}
\CommentTok{# Instead use the following code entered at the console}
\CommentTok{#}
\CommentTok{# rmarkdown::render(paste(d_home,"bin/snips/Report_SN_Summaries.R", sep = ""), "pdf_document")}
\CommentTok{#}
\NormalTok{#### }\RegionMarkerTok{END} \RegionMarkerTok{END} \RegionMarkerTok{END}\NormalTok{ ####}
\end{Highlighting}
\end{Shaded}


\end{document}
